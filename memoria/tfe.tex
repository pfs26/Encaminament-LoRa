\documentclass{tfgitic}[2024/07/01]
% « »
% Aqui carregueu packages complementaris que necessiteu
\usepackage{biblatex}
% \usepackage[table,xcdraw]{xcolor}
\usepackage{colortbl}
\usepackage{booktabs}
\usepackage{hhline}
\usepackage{caption}

\renewcommand{\figureautorefname}{figura}
\renewcommand{\tableautorefname}{taula}
\renewcommand{\sectionautorefname}{secció}
\renewcommand{\subsectionautorefname}{apartat}
\renewcommand{\subsubsectionautorefname}{subapartat}
\renewcommand{\chapterautorefname}{capítol}


% Indica quines bd bibliografiques usarem
\addbibresource{tfe.bib}

% \title{Disseny d’un protocol LoRa amb encaminament estàtic per aplicacions de monitoratge}
\title{Disseny d’un protocol LoRa amb encaminament estàtic per aplicacions de monitoratge}
\subtitle{Aplicació en xarxes de baix consum mitjançant sincronització temporal}
% \subtitle{Adaptació per a escenaris de baix consum mitjançant sincronització temporal}

% L'autor del treball. Admet gènere (vegeu 9.2 del manual) fent \author[f]{}
\author{Pol Flotats Sabata}

% La direcció. Un treball ordinari te un o excepcionalment dos directors.
% admet gènerer i número (vegeu 9.2 del manual)
\advisor{Jordi Bonet Dalmau i Arnau Arumi Casanovas}

% Si el treball es fa sota un conveni de pràctiques (modalitat empresa)
% llavors el director (advisor) és la persona de la empresa que
% dirigeix el treball i, a més, hi ha un professor que fa de tutor (counselor).
% En aquest cas també es consigna l'empresa (company)
% \counselor admet gènere (vegeu 9.2 del manual)
%
% \counselor{}
% \company{}

% Els àmbits temàtics en que es classifica el treball. Pregunteu al
% director.
\topics{}

% Si voleu dedicatòria descomenteu
%\dedication{}

% Si voleu agraïments descomenteu
%\begin{acknowledgments}
%\end{acknowledgments}


\begin{resum}
\end{resum}

\begin{abstract}
\end{abstract}




\begin{document}

% Si feu servir apèndixs, descomenteu
%\part{Memòria}

\chapter{Introducció}
\section{Objectius}
\section{Limitacions i abast del treball}
\section{Estructura de la memòria}

\chapter{Antecedents}
\section{Protocols de comunicació: capes i modularitat}
\label{sec:protocols}
Un protocol de comunicació defineix com dos o més dispositius d'una xarxa poden intercanviar informació. Ho fa mitjançant regles i normes que determinen la sintaxi ---format dels missatges---, la semàntica ---el seu significat---, i mecanismes de detecció i correcció d'errors.

Per tal que es pugui establir una comunicació, és necessari que els dispositius involucrats implementin el mateix protocol. Per facilitar-ho, es defineixen estàndards tècnics, publicats per organitzacions com l'\acro{iso} o l'\acro{ieee}, permetent que els fabricants puguin dissenyar dispositius compatibles. Un exemple de protocol estandaritzat és \acro{http}, amb el seu detall consultable a \cite{fielding_hypertext_2014}.

Per facilitar el disseny i implementació dels protocols, sovint es descomposen en protocols més simples. Aquests es poden agrupar en capes, on cada capa s'encarrega d'una part específica del procés de comunicació. El resultat és el que es coneix com una pila de protocols. 
En aquests dissenys, cada capa depèn de les capes inferiors per realitzar les seves funcions, i proporciona serveis a les capes superiors. El disseny i verificació de cada capa es pot fer de forma independent, i ofereixen la possibilitat d'implementar diferents protocols en cada capa, sempre que es mantingui la interfície definida entre elles.

Un dels models més coneguts és el model \acro{osi}, definit per set capes. Tot i no ser utilitzat en sistemes reals, és de gran utilitat en entorns acadèmics per comprendre la divisió per capes. Per a més informació, es pot consultar \cite{noauthor_isoiec_1994}.

En sistemes reals, el model més utilitzat és el model \acro{tcp/ip}, que estableix les bases d'Internet. Està format per quatre capes:
\begin{enumerate}
    \item \emph{Enllaç}. És la capa de més baix nivell. S'encarrega de la comunicació entre dispositius d'una mateixa xarxa ---és a dir, es poden comunicar directament---, i de la detecció i correció d'errors produits en la comunicació. També gestiona l'accés al medi físic per on es transmeten les dades, que sovint és compartit amb altres dispositius.
    \item \emph{Xarxa}. Gestiona la comunicació entre dispositius que es troben en xarxes diferents i, per tant, no es poden comunicar directament. Per fer-ho, s'utilitzen dispositius intermedis, coneguts com a encaminadors (\est{routers}), que determinen la ruta més eficient per fer arribar les dades al seu destí.
    \item \emph{Transport}. Defineix la connexió d'extrem a extrem entre els dispositius origen i destí i, si és necessari, que la transmissió sigui fiable. Els protocols d'aquesta capa poden oferir mecanismes com l'ordenació de missatges, l'eliminació de missatges duplicats i la gestió de congestió. Els dos protocols més coneguts d'aquesta capa són \acro{tcp}, que garanteix la transmissió fiable de dades mitjançant confirmacions i retransmissions, i \acro{udp}, que no garanteix fiabilitat, però és més eficient per aplicacions en temps real com el contingut en estríming o videjocs.
    \item \emph{Aplicació}. És la capa més alta i propera a l'usuari final. Defineix els protocols que utilitzen les aplicacions per comunicar-se a través de la xarxa, com ara \acro{http}, utilitzat per a la navegació web, o protocols de suport, com ara \acro{dns}, per a la resolució de noms de domini.
\end{enumerate}
Per a l'estàndard complet i més detall, es pot consultar \cite{braden_requirements_1989}. 

% \subsection{Referència al model TCP/IP}
\section{Tecnologia LoRa i LoRaWAN}
Els termes \emph{LoRa} i \emph{LoRaWAN} sovint es confonen i s'utilitzen ambdós termes indistintament. Mentre que el primer és una tecnologia propietària, el segon és mantingut per una associació sense ànim de lucre. En aquest apartat es presenten ambdues tecnologies, indicant-ne les seves característiques i limitacions.

\subsection{LoRa}
El nom de \emph{LoRa} prové de l'anglès \est{Long Range}. És una tecnologia de comunicació sense fils, de llarg abast, i de baix consum, fent que sigui especialment útil per a aplicacions d'Internet de les Coses (\acro{iot}). Va ser desenvolupada per \emph{Cycleo}, que va ser adquirida per \emph{Semtech} el 2012.

Utilitza freqüència sub-GHz en bandes \acro{ism} ---\emph{Industrial, Scientific and Medical}---, que són bandes de freqüència lliures de llicència. Això permet que qualsevol persona pugui utilitzar la tecnologia sense necessitat d'obtenir una llicència. Malgrat això, existeixen limitacions legals, com la potència màxima de transmissió o la quantitat de dades que es poden transmetre en un període de temps determinat. Per a més informació, es pot consultar \ref{subsec:limitacions_legals}.

Per assolir una comunicació de llarg abast, LoRa utilitza modulació de \emph{chirp spread spectrum} (\acro{css}), que permet una comunicació robusta i fiable en entorns amb interferències. Aquesta modulació es basa en codificar cada símbol com un senyal que s'escampa per tot l'ample de banda disponible, augmentant (o disminuint) de forma lineal la freqüència al llarg del temps. Gràcies a que la freqüència varia de forma lineal, també és resistent a l'efecte \emph{Doppler}, com s'ha estudiat a \cite{doroshkin_experimental_2019}. Una representació visual del funcionament de la modulació \acro{css} es pot veure a \cite{richard_wenner_lora_2017}.

Consta de diversos paràmetres que es poden ajustar per adaptar la comunicació a diferents necessitats:
\begin{itemize}
    \item \emph{Ample de banda}. És l'ample de banda (\acro{bw}) del canal de transmissió, sovint de \SI{125}{\kHz}, però també pot ser de \SI{250}{\kHz} o \SI{500}{\kHz}. Un ample de banda més gran permet una major velocitat de transmissió, però redueix la sensibilitat de recepció i, per tant, el rang de comunicació.
    \item \emph{Data rate}. És la velocitat de transmissió de dades (\acro{dr}), que depèn de l'ample de banda i del factor d'espargiment. Un data rate més alt permet una major velocitat de transmissió, però redueix la sensibilitat de recepció i, per tant, el rang de comunicació.
    \item \emph{Spreading Factor}. El factor d'espargiment (\acro{sf}) defineix la durada dels símbols i, per tant, la velocitat de transmissió. Un \acro{sf} més alt permet una major sensibilitat de recepció, ja que cada símbol ocupa més temps i, per tant, és més fàcil de detectar. Això permet una comunicació a llarg abast, però redueix la velocitat de transmissió. Els valors possibles de \acro{sf} són entre 7 i 12 i, per cada increment de 1 d'\acro{sf}, es redueix la velocitat de transmissió a la meitat. Una característica molt important és que els diferents \acro{sf} són ortogonals entre ells. Això implica que senyals modulats amb diferents \acro{sf} i en un mateix canal no interfereixin entre ells. 
    \item \emph{Coding Rate}. La taxa de codificació (\acro{cr}) defineix la proporció d'informació útil que es transmet en comparació amb la quantitat total de dades enviades. Els possibles valors són $\frac{4}{5}$, $\frac{4}{6}$, $\frac{4}{7}$ i $\frac{4}{8}$. Així, per un \acro{cr} de $\frac{4}{5}$, per cada quatre bits d'informació útil, se n'afegeix un de redundància.
\end{itemize}

Tots aquests paràmetres estan relacionats entre ells, i és important trobar un equilibri entre ells per aconseguir la millor comunicació possible. Per exemple, si es vol obtenir una comunicació a llarg abast, es pot augmentar el \acro{sf} i el \acro{cr}. Això reduirà la velocitat de transmissió i, per tant, augmentarà el consum energètic, ja que serà necessari estar transmetent més temps. En canvi, si es vol una comunicació ràpida, es pot reduir el \acro{sf} i augmentar el \acro{bw}, però això reduirà el rang de comunicació.

Amb una configuració adequada, es poden aconseguir comunicacions de fins a quinze quilòmetres en entorns rurals, i amb una vida útil de les bateries de diversos mesos i anys, com especifica Semtech a \cite{noauthor_lora_2024}.

\subsection{LoRaWAN}
Pel que fa a LoRaWAN, es tracta d'una definició de l'arquitectura d'un sistema \acro{LPWAN} (\est{Low Power Wide Area Network}), desenvolupada i mantenida per la \href{https://lora-alliance.org}{LoRa Alliance}. Defineix el protocol de comunicació i l'arquitectura de la xarxa, així com els mecanismes de seguretat i gestió de la xarxa. 

Utilitzant l'estructura per capes vista a la \autoref{sec:protocols}, LoRaWAN opera principalment a nivell d'enllaç, malgrat que també implementa característiques d'altres capes (com ara el control de sessions). A nivell físic, utilitza la modulació LoRa.

Pel que fa a l'estructura de la xarxa, consta de 4 components principals:
\begin{itemize}
    \item \emph{Dispositius finals}. Sensors o actuadors que transmeten, a través de LoRa, missatges als \est{gateways}, conegut com a \est{uplinks}. També pot ser a l'inrevés, on els \est{gateways} transmeten missatges als dispositius finals, conegut com a \est{downlinks}.  
    \item \emph{Gateways}. Dispositius que reben els missatges dels dispositius finals i els transmeten al servidor de la xarxa. La comunicació entre els \est{gateways} i el servidor de la xarxa es realitza a través d'Internet, i no utilitza LoRa. 
    \item \emph{Servidor de la xarxa}. Dispositiu que gestiona els dispositius finals, els \est{gateways} i aplicacions de la xarxa. S'encarrega de la gestió de les comunicacions, la seguretat i la configuració dels dispositius finals. Gestionen també el xifrat punt a punt ---entre dispositius finals i servidors d'aplicació---.
    \item \emph{Servidor d'aplicació}. Processa les dades d'aplicació enviades pels dispositius finals.
\end{itemize}
Aquests dispositius es troben organitzats en una estructura d'«estrella d'estrelles», on a la part central hi trobem el servidor de la xarxa. Aquest es comunica amb múltiples \est{gateways}, i cada \est{gateway} es comunica a la vegada amb múltiples dispositius finals. 
Un fet important és que un dispositiu final no escull el \est{gateway} amb qui comunicar-se: en transmetre, tots els \est{gateways} que reben les dades ho reenvien al servidor de la xarxa. El servidor de xarxa és qui s'encarrega de detecar missatges duplicats i escollir quin és el millor \est{gateway} per transmetre un \est{downlink}. Aquesta estructura simplifica el disseny dels dispositius finals, amb el cost d'una major complexitat en el servidor de xarxa, tenint en compte que hi ha un únic servidor de xarxa per una quantitat N de dispositius finals.

Els dispositius finals poden operar en tres modes diferents, coneguts com a \emph{classes}:
\begin{itemize}
    \item \emph{Classe A}. És la classe més senzilla i de menor consum. Els dispositius finals només poden rebre missatges després d'haver realitzat una transmissió (\est{uplink}), moment en el qual obren dues finestres de recepció per rebre \est{downlinks}. Aquesta és la classe més utilitzada en aplicacions d'\acro{iot}, ja que permet un funcionament de baix consum, sent únicament necessari posar-se en mode de recepció després de fer una transmissió.
    \item \emph{Classe B}. Els dispositius finals obren finestres de recepció en moments predeterminats (coneguts com a \est{ping slots}), a més de les finestres de recepció que obren després d'un \est{uplink}. Aquest fet fa que la latència de recepció de \est{downlinks} sigui molt menor que en la classe A.
    \item \emph{Classe C}. Obren també dues finestres de recepció després d'un \est{uplink}, amb la diferència que la segona finestra es manté oberta fins la següent transmissió. Així, es pot considerar que aquests dispositius sempre estan rebent, amb l'exepció de quan transmeten. La latència de recepció de \est{downlinks} és la mínima d'entre les tres classes, però el consum energètic és màxim.
\end{itemize}

Una altra característica important de LoRaWAN és la seva capacitat d'autoajustament, conegut com a \acro{adr} (\est{Adaptative Data Rate}). Els dispositius finals poden ajustar automàticament els paràmetres de comunicació, com el \acro{sf} o la potència de transmissió, per adaptar-se a les condicions del canal de comunicació. Quan s'utilitza, el servidor de la xarxa pot indicar als dispositius finals quins paràmetres utilitzar, reduint el consum energètic i possibles interferències amb altres dispositius. Per exemple, un dispositiu molt proper a un \est{gateway} hauria d'utilitzar \acro{sf} més baixos, mentre que dispositius llunyans n'haurien d'utilitzar de més elevats.

\subsection{The Things Network}
Es tracta d'un projecte d'\acro{iot} que proporciona un servidor de xarxa públic i gratuït. El seu objectiu és crear una xarxa global de dispositius LoRaWAN, permetent la comunicació entre ells sense necessitat d'infraestructura pròpia. Els usuaris poden connectar els seus \est{gateways} a la xarxa i compartir la seva cobertura amb altres usuaris.

També ofereix la possibilitat d'afegir dispositius finals a la xarxa, i gestionar-los a través de la seva interfície web. A més, permeten la integració amb altres serveis, com ara \emph{Node-RED} o \emph{Grafana}, per visualitzar i analitzar les dades dels dispositius finals.

\subsection{Limitacions legals}
\label{subsec:limitacions_legals}
Com s'ha comentat anteriorment, LoRa utilitza bandes de freqüència lliures de llicència. Per garantir un ús adequat d'aquestes bandes, cada país estableix una sèrie de limitacions legals. Aquestes limitacions poden variar d'un país a un altre, i és important tenir-les en compte a l'hora de dissenyar un sistema de comunicació LoRa. 

Pel que fa a la Unió Europea, l'entitat responsable de la regulació de les comunicacions és l'\acro{etsi} (\est{European Telecommunications Standards Institute}). Estableix l'ús de LoRa en les freqüències entre \SI{863}{\MHz} i \SI{870}{\MHz}, amb un ample de banda màxim de \SI{250}{\kHz}.

A més, estableix un límit de potència màxima de transmissió, i un màxim d'utilització del canal. Aquestes limitacions depenen de la sub-banda utilitzada (dins el rang anteriorment mencionat), però de forma genèrica es considera una potència màxima de \SI{16}{\deci\bel\text{m}}, i una utilització màxima del canal (conegut com a \emph{duty cyle}) de l'\SI{1}{\%}. Així, per cada hora de funcionament, poden transmetre un màxim de 36 segons. Tots els detalls sobre les limitacions legals a la Unió Europea es poden consultar a \cite{etsi_etsi_nodate}.

Si s'utilitza un servidor de xarxa públic com \emph{The Things Network}, és important tenir en compte que aquest també pot aplicar limitacions addicionals. Pel que fa a \acro{ttn}, estableix una política d'ús just (\emph{fair use policy}) que limita l'ús del canal en 30 segons per dispositiu final i dia. A més, s'estableix un límit de deu \est{downlinks} per dispositiu i dia. És important tenir aquestes limitacions en compte en dissenyar un sistema; en cas d'incumpliment, el servidor de xarxa podria bloquejar l'accés del dispositiu final.

\section{Treballs relacionats}
Existeixen diversos projectes amb un objectiu similar al d'aquest treball, com ara \emph{Meshtastic} o \emph{LoRaMesher}. Tots dos projectes utilitzen LoRa per crear xarxes de comunicació entre dispositius finals, però presenten diferencies amb els objectius d'aquest treball. A més, consten també de limitacions, a les qual se'ls vol donar solució. En aquest apartat es presenten ambdós projectes, amb les seves característiques i limitacions, així com diferències vers el treball presentat. 

\subsection{Meshtastic}
Es tracta d'un projecte de codi obert que té com a objectiu crear una xarxa de comunicació descentralitzada, i pensada per a ser utilitzada en dispositius de baix consum i cost. El seu repositori es troba disponible a \href{https://github.com/meshtastic}{GitHub}.

Utilitza comunicació LoRa per a transmetre missatges entre dispositius, evitant dependre d'altres infraestructures com internet. Per tal d'aconseguir-ho, tots els dispositius actuen com a encaminadors: quan reben un missatge que no està destinat a ells, el reenvien, permetent que altres dispositius puguin repetir l'encaminament fins arribar al destí final. Una característica molt interessant d'aquest projecte és que ofereix aplicacions mòbils i web per interactuar amb els dispositius de la xarxa, permetent enviar missatges de forma molt senzilla.

L'estratègia d'encaminament que utilitza es basa en «inundació» (\est{flooding}), on cada dispositiu reenvia tots els missatges que rep, sempre que no sigui ell el destinetari, sense tenir en compte si ja els ha rebut o no. Aquest disseny tampoc considera la ubicació del destí final, fent que el missatge es transmeti per tota la xarxa. Aquests fets poden provocar un sobreús del canal, i un augment del consum energètic dels dispositius. 

Per intentar reduir aquest efecte, implementa un estil d'«inundació controlada». Quan un dispositiu rep un missatge, espera un temps proporcional a la relació senyal soroll (\acro{snr}) del missatge rebut abans de retransmetre, i cance\l.la l'enviament si un node realitza la transmissió abans. Així, un dispositiu més proper a l'origen (major \acro{snr}) esperarà més temps que un de llunyà i, per tant, el missatge el retransmetrà únicament el dispositiu llunyà. És cert que això redueix el nombre de transmissions que realitza cada dispositiu, però no evita que el missatge es retransmeti per tota la xarxa. 

Un altre inconvenient és que, a causa que els dispositius no saben quan altres dispositius realitzaran una transmissió, hagin d'estar sempre actius i en mode de recepció. Això provoca que el consum energètic dels dispositius sigui elevat (de l'ordre dels \SI{10}{\milli\watt}), fent-los poc adequats per aplicacions de monitoratge.

Durant el desenvolupament d'aquest treball, Meshtastic ha iniciat la implementació d'un protocol d'encaminament per a missatges entre dos únics dispositius. La idea d'aquest protocol és descobrir la ruta més òptima entre dos dispositius utilitzant el mecanisme d'«inundació controlada»; un cop el destí rep el missatge, respon amb un nou missatge amb la informació de la ruta, que conté tots els dispositius que han de retransmetre el missatge per arribar fins a ell. En el procés de fer arribar aquesta resposta al transmissor del missatge, tots els nodes intermedis poden generar la seva taula de rutes. En les següents transmissions, si es coneix la ruta, els missatges els reenvien únicament els nodes de la ruta. Els detalls de la implementació i funcionament d'aquest protocol es poden consultar a \cite{open_source_mesh_nodate}.

\section{LoRaMesher}
Es tracta d'una llibreria que implementa un protocol d'encaminament per a ser utilitzat en xarxes basades en LoRa. 

Cada dispositiu consta d'una taula d'encaminament, on es defineixen quin és el següent dispositiu a qui han d'enviar un missatge per fer-lo arribar al destí final. De forma periòdica, cada node transmet informació sobre la seva taula d'encaminament; els nodes que es troben a l'abast reben el missatge i actualitzen la seva taula d'encaminament amb la informació rebuda, no només coneixent informació sobre els seus veïns directes, sinó també sobre els veïns dels seus veïns, i així successivament. Es poden consultar els detalls d'aquesta implementació a \cite{sole_implementation_2022}.

Aquesta solució resol el problema de la «inundació controlada» presentat anteriorment, ja que únicament els nodes que es troben a la ruta reenvien el missatge. Malgrat això, presenta un dels inconvenients vists prèviament: tots els dispositius han d'estar sempre actius i en mode de recepció. Com s'ha mencionat, això provoca un consum elevat, reduint la viabilitat d'ús en aplicacions de monitoratge.

\chapter{Desenvolupament del protocol de comunicació}
\section{Disseny per capes}
\subsection{Capa física}
\subsection{Accés al medi}
\subsection{Encaminament estàtic}
\subsection{Capa de transport}
\subsection{Capa d’aplicació}
\section{Implementació}
\subsection{Arquitectura del codi i comunicació entre capes}
\subsection{Biblioteques i entorn de desenvolupament}
\subsection{Proves i validació funcional}

\chapter{Adaptació a entorns de baix consum}
\section{Descripció de l’escenari: xarxa lineal de sensors}
\section{Limitacions del protocol sense optimitzacions energètiques}
% \section{Estratègies de sincronització i activació temporal}
\section{Estratègies de sincronització}
\subsection{Sincronització explícita}
\subsection{Sincronització implícita}
% \subsection{Avantatges i inconvenients de cada enfocament}
% \subsection{Compromisos entre consum i generalitat del protocol}

\chapter{Conclusions}
% \section{Resultats assolits}
% \section{Limitacions del disseny final}
% \section{Punts clau de millora}

\chapter{Treball futur}
% \section{Suport a topologies més complexes}
% \section{Optimització del consum energètic}
% \section{Integració amb LoRaWAN o altres protocols}


% \chapter{Introducció}
% \label{chap:intro}
% \section{Objectius}
% \label{sec:objectius}
% \section{Limitacions del treball}
% \section{Organització de la memòria}

% \chapter{Antecedents}
% \section{Introducció a LoRa i LoRaWAN}
% % Mencionar limitacions legals de TTN i EU868
% \section{Sobre els protocols de comunicació}
% \section{Treballs relacionats}
% % meshtastic -> xarxa mesh però sense baix consum, broadcasts podent afectar communicació, etc.
% % tampoc permeten compatibilitat amb lorawan
% % \section{Solucions existents}
% % loramesher -> sense broadcast, però requereixen descobriment de rutes previs
% % Cap dels dos permeten baix consum i modes de deep sleep

% % Aqui van els capitols especifics del treball
% \chapter{Disseny del protocol de comunicació} % no es veu afectat per material
% \section{Medi físic}
% \subsection{LoRa}
% \subsection{LoRaWAN}
% % Uplinks es faran amb ACK o no depenent de define en compilar
% % fet així per evitar que s'hagin de fer molts downlinks per ACKs
% % ja que TTN limita (fair use policy) a 10 downlinks per dia
% \section{Accés al medi}
% % Mencionar que seria aquí on es faria limitació d'accés al medi
% % Podent afegir un últim estat que "esperi" un temps fins següent TX
% % Parlar llavors sobre les possibles limitacions que això podria comportar
% % (per exemple si es limiten ACKs de capa MAC, o no es limiten, o es limiten ACKs de transport...)
% \section{Encaminament}
% \section{Transport}
% \section{Aplicació}
% \chapter{Implementació del protocol}
% \section{Estructura}
% \subsection{Organització del codi}
% % S'ha implementat seguint model de capes, on cada capa és un mòdul.
% % Cada capa inferior notifica a la superior a través de callbacks, 
% % i cada capa superior notifica a la inferior a través de mètodes.
% % S'ha intentat que no hi hagi mètodes bloquejants, amb excepció dels 
% % mètodes "finals" com transmissió.
% \subsection{Biblioteques i entorn de desenvolupament}
% % platformio + radiolib (comentar radiohead + lmic)
% \section{Protocol de comunicació}
% \subsection{Capa física}
% \subsection{Accés al medi}
% \subsection{Encaminament}
% \subsection{Transport}
% \subsection{Aplicació}
% \section{Validació del funcionament i resultats}

% \chapter{Aplicació en entorns de baix consum}
% % explicar avantatges i inconvenients d'utilitzar
% % el protocol de sincronització genèric aqui. També s'hauria d'explicar el seu disseny

% % quines son totes les opcions plantejades, avantatges i inconvenients, i perquè
% \section{Descripció de la situació}
% % Com és la xarxa, i què es vol aconseguir

% % Per a cada iteració: avantatges i inconvenients. Què implicaria afegir nous nodes?
% % Limitacions de temps? memòria?

% \chapter{Conclusions}

% \chapter{Treball futur}

% https://www.latex-tables.com/
% configurar amb opció de SCALE
\begin{figure}
    \centering
    \resizebox{\linewidth}{!}{%
        \begin{tabular}{l||cccccccccccccccccc} 
            \hhline{~|t|~~~~~~~----~~~~~~~}
            \textbf{N1} &                          &                            &                            &                            &                            &                            & \multicolumn{1}{c|}{}      & \multicolumn{1}{c|}{RX2:9} & \multicolumn{1}{c|}{TX1:9} & \multicolumn{1}{c|}{{\cellcolor[rgb]{0.753,0.749,0.737}}RXG} & \multicolumn{1}{c|}{{\cellcolor[rgb]{0.753,0.749,0.737}}TXG} &                                                              &                                                              &                          &                          &                          &                          &                           \\ 
            \hhline{~||~~~~~~------~~~~~~}
            \textbf{N2} &                          &                            &                            &                            &                            & \multicolumn{1}{c|}{}      & \multicolumn{1}{c|}{RX3:9} & \multicolumn{1}{c|}{TX2:9} &                            & \multicolumn{1}{c|}{}                                        & \multicolumn{1}{c|}{{\cellcolor[rgb]{0.753,0.749,0.737}}RXG} & \multicolumn{1}{c|}{{\cellcolor[rgb]{0.753,0.749,0.737}}TXG} &                                                              &                          &                          &                          &                          &                           \\ 
            \hhline{~||~~~~~---~~---~~~~~}
            \textbf{N3} &                          &                            &                            &                            & \multicolumn{1}{c|}{}      & \multicolumn{1}{c|}{RX4:9} & \multicolumn{1}{c|}{TX3:9} &                            &                            &                                                              & \multicolumn{1}{c|}{}                                        & \multicolumn{1}{c|}{{\cellcolor[rgb]{0.753,0.749,0.737}}RXG} & \multicolumn{1}{c|}{{\cellcolor[rgb]{0.753,0.749,0.737}}TXG} &                          &                          &                          &                          &                           \\ 
            \hhline{~||~~~~---~~~~---~~~~}
            \textbf{N4} &                          &                            &                            & \multicolumn{1}{c|}{}      & \multicolumn{1}{c|}{RX5:9} & \multicolumn{1}{c|}{TX4:9} &                            &                            &                            &                                                              &                                                              & \multicolumn{1}{c|}{}                                        & \multicolumn{1}{c|}{{\cellcolor[rgb]{0.753,0.749,0.737}}RXG} & \multicolumn{1}{c|}{TXG} &                          &                          &                          &                           \\ 
            \cline{5-7}\cline{14-16}
            \textbf{N5} &                          &                            & \multicolumn{1}{c|}{}      & \multicolumn{1}{c|}{RX6:9} & \multicolumn{1}{c|}{TX5:9} &                            &                            &                            &                            &                                                              &                                                              &                                                              & \multicolumn{1}{c|}{}                                        & \multicolumn{1}{c|}{RXG} & \multicolumn{1}{c|}{TXG} &                          &                          &                           \\ 
            \cline{4-6}\cline{15-17}
            \textbf{N6} &                          & \multicolumn{1}{c|}{}      & \multicolumn{1}{c|}{RX7:9} & \multicolumn{1}{c|}{TX6:9} &                            &                            &                            &                            &                            &                                                              &                                                              &                                                              &                                                              & \multicolumn{1}{c|}{}    & \multicolumn{1}{c|}{RXG} & \multicolumn{1}{c|}{TXG} &                          &                           \\ 
            \cline{3-5}\cline{16-18}
            \textbf{N7} & \multicolumn{1}{c|}{}    & \multicolumn{1}{c|}{RX8:9} & \multicolumn{1}{c|}{TX7:9} &                            &                            &                            &                            &                            &                            &                                                              &                                                              &                                                              &                                                              &                          & \multicolumn{1}{c|}{}    & \multicolumn{1}{c|}{RXG} & \multicolumn{1}{c|}{TXG} &                           \\ 
            \cline{2-4}\cline{17-19}
            \textbf{N8} & \multicolumn{1}{c|}{RX9} & \multicolumn{1}{c|}{TX8:9} &                            &                            &                            &                            &                            &                            &                            &                                                              &                                                              &                                                              &                                                              &                          &                          & \multicolumn{1}{c|}{}    & \multicolumn{1}{c|}{RXG} & \multicolumn{1}{c|}{TXG}  \\ 
            \cline{2-3}\cline{18-19}
            \textcolor[rgb]{0.1,1,0.2}{\textbf{N9}} & \multicolumn{1}{c|}{TX9} &                            &                            &                            &                            &                            &                            &                            &                            &                                                              &                                                              &                                                              &                                                              &                          &                          &                          & \multicolumn{1}{c|}{}    & \multicolumn{1}{c|}{RXG}  \\
            \cline{2-2}\cline{19-19}
        \end{tabular}
    }
    \caption{Model sense missatge de sincronització, amb dades acumulatives}
    \label{fig:noSyncAcumulatiu}
\end{figure}

\autoref{chap:intro}
\autoref{sec:objectius}
\autoref{tab:noSyncAcumulatiu}

\listoftables
\listoffigures

\printbibliography


% Si feu servir apèndixs, descomenteu
% (també la  \part del principi del document)
%\appendix
%\part{Apèndixs}
%\chapter{Un apèndix}


\end{document}

%%% Local Variables:
%%% mode: latex
%%% TeX-master: t
%%% LaTeX-biblatex-use-Biber: t
%%% End:
