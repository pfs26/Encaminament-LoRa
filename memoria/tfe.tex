\documentclass{tfgitic}[2024/07/01]

% Aqui carregueu packages complementaris que necessiteu
\usepackage{biblatex}
% \usepackage[table,xcdraw]{xcolor}
\usepackage{colortbl}
\usepackage{booktabs}
\usepackage{hhline}
\usepackage{caption}

\renewcommand{\figureautorefname}{figura}
\renewcommand{\tableautorefname}{taula}
\renewcommand{\sectionautorefname}{secció}
\renewcommand{\subsectionautorefname}{apartat}
\renewcommand{\subsubsectionautorefname}{subapartat}
\renewcommand{\chapterautorefname}{capítol}


% Indica quines bd bibliografiques usarem
\addbibresource{tfe.bib}

\title{Disseny de nodes LoRa amb encaminament estàtic per aplicacions de monitoratge}
% Si voleu subtitol descomenteu
\subtitle{Viabilitat d'ús en entorns de baix consum}

% L'autor del treball. Admet gènere (vegeu 9.2 del manual) fent \author[f]{}
\author{Pol Flotats Sabata}

% La direcció. Un treball ordinari te un o excepcionalment dos directors.
% admet gènerer i número (vegeu 9.2 del manual)
\advisor{Jordi Bonet Dalmau i Arnau Arumi Casanovas}

% Si el treball es fa sota un conveni de pràctiques (modalitat empresa)
% llavors el director (advisor) és la persona de la empresa que
% dirigeix el treball i, a més, hi ha un professor que fa de tutor (counselor).
% En aquest cas també es consigna l'empresa (company)
% \counselor admet gènere (vegeu 9.2 del manual)
%
% \counselor{}
% \company{}

% Els àmbits temàtics en que es classifica el treball. Pregunteu al
% director.
\topics{}

% Si voleu dedicatòria descomenteu
%\dedication{}

% Si voleu agraïments descomenteu
%\begin{acknowledgments}
%\end{acknowledgments}


\begin{resum}
\end{resum}

\begin{abstract}
\end{abstract}




\begin{document}

% Si feu servir apèndixs, descomenteu
%\part{Memòria}

\chapter{Introducció}
\label{chap:intro}
\section{Objectius}
\label{sec:objectius}
\section{Limitacions del treball}
\section{Organització de la memòria}

\chapter{Antecedents}
\section{Introducció a LoRa i LoRaWAN}
\section{Sobre els protocols de comunicació}
\section{Treballs relacionats}
% meshtastic -> xarxa mesh però sense baix consum, broadcasts podent afectar communicació, etc.
% tampoc permeten compatibilitat amb lorawan
% \section{Solucions existents}
% loramesher -> sense broadcast, però requereixen descobriment de rutes previs
% Cap dels dos permeten baix consum i modes de deep sleep

% Aqui van els capitols especifics del treball
\chapter{Disseny del protocol de comunicació} % no es veu afectat per material
\section{Medi físic}
\subsection{LoRa}
\subsection{LoRaWAN}
% Uplinks es faran amb ACK o no depenent de define en compilar
% fet així per evitar que s'hagin de fer molts downlinks per ACKs
% ja que TTN limita (fair use policy) a 10 downlinks per dia
\section{Accés al medi}
\section{Encaminament}
\section{Transport}
\section{Aplicació}

\chapter{Implementació del protocol}
\section{Estructura}
\subsection{Organització del codi}
% S'ha implementat seguint model de capes, on cada capa és un mòdul.
% Cada capa inferior notifica a la superior a través de callbacks, 
% i cada capa superior notifica a la inferior a través de mètodes.
% S'ha intentat que no hi hagi mètodes bloquejants, amb excepció dels 
% mètodes "finals" com transmissió.
\subsection{Biblioteques i entorn de desenvolupament}
% platformio + radiolib (comentar radiohead + lmic)
\section{Protocol de comunicació}
\subsection{Capa física}
\subsection{Accés al medi}
\subsection{Encaminament}
\subsection{Transport}
\subsection{Aplicació}
\section{Validació del funcionament i resultats}

\chapter{Aplicació en entorns de baix consum}
% explicar avantatges i inconvenients d'utilitzar
% el protocol de sincronització genèric aqui. També s'hauria d'explicar el seu disseny

% quines son totes les opcions plantejades, avantatges i inconvenients, i perquè

\chapter{Conclusions}

\chapter{Treball futur}

% https://www.latex-tables.com/
% configurar amb opció de SCALE
\begin{figure}
    \centering
    \resizebox{\linewidth}{!}{%
        \begin{tabular}{l||cccccccccccccccccc} 
            \hhline{~|t|~~~~~~~----~~~~~~~}
            \textbf{N1} &                          &                            &                            &                            &                            &                            & \multicolumn{1}{c|}{}      & \multicolumn{1}{c|}{RX2:9} & \multicolumn{1}{c|}{TX1:9} & \multicolumn{1}{c|}{{\cellcolor[rgb]{0.753,0.749,0.737}}RXG} & \multicolumn{1}{c|}{{\cellcolor[rgb]{0.753,0.749,0.737}}TXG} &                                                              &                                                              &                          &                          &                          &                          &                           \\ 
            \hhline{~||~~~~~~------~~~~~~}
            \textbf{N2} &                          &                            &                            &                            &                            & \multicolumn{1}{c|}{}      & \multicolumn{1}{c|}{RX3:9} & \multicolumn{1}{c|}{TX2:9} &                            & \multicolumn{1}{c|}{}                                        & \multicolumn{1}{c|}{{\cellcolor[rgb]{0.753,0.749,0.737}}RXG} & \multicolumn{1}{c|}{{\cellcolor[rgb]{0.753,0.749,0.737}}TXG} &                                                              &                          &                          &                          &                          &                           \\ 
            \hhline{~||~~~~~---~~---~~~~~}
            \textbf{N3} &                          &                            &                            &                            & \multicolumn{1}{c|}{}      & \multicolumn{1}{c|}{RX4:9} & \multicolumn{1}{c|}{TX3:9} &                            &                            &                                                              & \multicolumn{1}{c|}{}                                        & \multicolumn{1}{c|}{{\cellcolor[rgb]{0.753,0.749,0.737}}RXG} & \multicolumn{1}{c|}{{\cellcolor[rgb]{0.753,0.749,0.737}}TXG} &                          &                          &                          &                          &                           \\ 
            \hhline{~||~~~~---~~~~---~~~~}
            \textbf{N4} &                          &                            &                            & \multicolumn{1}{c|}{}      & \multicolumn{1}{c|}{RX5:9} & \multicolumn{1}{c|}{TX4:9} &                            &                            &                            &                                                              &                                                              & \multicolumn{1}{c|}{}                                        & \multicolumn{1}{c|}{{\cellcolor[rgb]{0.753,0.749,0.737}}RXG} & \multicolumn{1}{c|}{TXG} &                          &                          &                          &                           \\ 
            \cline{5-7}\cline{14-16}
            \textbf{N5} &                          &                            & \multicolumn{1}{c|}{}      & \multicolumn{1}{c|}{RX6:9} & \multicolumn{1}{c|}{TX5:9} &                            &                            &                            &                            &                                                              &                                                              &                                                              & \multicolumn{1}{c|}{}                                        & \multicolumn{1}{c|}{RXG} & \multicolumn{1}{c|}{TXG} &                          &                          &                           \\ 
            \cline{4-6}\cline{15-17}
            \textbf{N6} &                          & \multicolumn{1}{c|}{}      & \multicolumn{1}{c|}{RX7:9} & \multicolumn{1}{c|}{TX6:9} &                            &                            &                            &                            &                            &                                                              &                                                              &                                                              &                                                              & \multicolumn{1}{c|}{}    & \multicolumn{1}{c|}{RXG} & \multicolumn{1}{c|}{TXG} &                          &                           \\ 
            \cline{3-5}\cline{16-18}
            \textbf{N7} & \multicolumn{1}{c|}{}    & \multicolumn{1}{c|}{RX8:9} & \multicolumn{1}{c|}{TX7:9} &                            &                            &                            &                            &                            &                            &                                                              &                                                              &                                                              &                                                              &                          & \multicolumn{1}{c|}{}    & \multicolumn{1}{c|}{RXG} & \multicolumn{1}{c|}{TXG} &                           \\ 
            \cline{2-4}\cline{17-19}
            \textbf{N8} & \multicolumn{1}{c|}{RX9} & \multicolumn{1}{c|}{TX8:9} &                            &                            &                            &                            &                            &                            &                            &                                                              &                                                              &                                                              &                                                              &                          &                          & \multicolumn{1}{c|}{}    & \multicolumn{1}{c|}{RXG} & \multicolumn{1}{c|}{TXG}  \\ 
            \cline{2-3}\cline{18-19}
            \textbf{N9} & \multicolumn{1}{c|}{TX9} &                            &                            &                            &                            &                            &                            &                            &                            &                                                              &                                                              &                                                              &                                                              &                          &                          &                          & \multicolumn{1}{c|}{}    & \multicolumn{1}{c|}{RXG}  \\
            \cline{2-2}\cline{19-19}
        \end{tabular}
    }
    \caption{Model sense missatge de sincronització, amb dades acumulatives}
    \label{tab:noSyncAcumulatiu}
\end{figure}

\autoref{chap:intro}
\autoref{sec:objectius}
\autoref{tab:noSyncAcumulatiu}

\listoftables
\listoffigures
\printbibliography


% Si feu servir apèndixs, descomenteu
% (també la  \part del principi del document)
%\appendix
%\part{Apèndixs}
%\chapter{Un apèndix}


\end{document}

%%% Local Variables:
%%% mode: latex
%%% TeX-master: t
%%% LaTeX-biblatex-use-Biber: t
%%% End:
