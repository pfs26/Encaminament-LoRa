\documentclass{tfgitic}[2025/06/06]

% Aqui carregueu packages complementaris que necessiteu
%\usepackage{}

% Indica quines bd bibliografiques usarem
\addbibresource{tfe.bib}


\title{Disseny de nodes LoRa amb encaminament estàtic per aplicacions de monitoratge}
% Si voleu subtitol descomenteu
%\subtitle{}

% L'autor del treball. Admet gènere (vegeu 9.2 del manual) fent \author[f]{}
\author{Pol Flotats Sabata}

% La direcció. Un treball ordinari te un o excepcionalment dos directors.
% admet gènerer i número (vegeu 9.2 del manual)
\advisor{Jordi Bonet Dalmau i Arnau Arumi Casanovas}

% Si el treball es fa sota un conveni de pràctiques (modalitat empresa)
% llavors el director (advisor) és la persona de la empresa que
% dirigeix el treball i, a més, hi ha un professor que fa de tutor (counselor).
% En aquest cas també es consigna l'empresa (company)
% \counselor admet gènere (vegeu 9.2 del manual)
%
% \counselor{}
% \company{}

% Els àmbits temàtics en que es classifica el treball. Pregunteu al
% director.
\topics{}

% Si voleu dedicatòria descomenteu
%\dedication{}

% Si voleu agraïments descomenteu
%\begin{acknowledgments}
%\end{acknowledgments}


\begin{resum}
\end{resum}

\begin{abstract}
\end{abstract}




\begin{document}

% Si feu servir apèndixs, descomenteu
%\part{Memòria}

\chapter{Introducció}
\section{Objectius}
\section{Limitacions del treball}
\section{Organització de la memòria}

\chapter{Antecedents}
\section{Introducció a LoRa i LoRaWAN}
\section{Treballs relacionats}
% meshtastic -> xarxa mesh però sense baix consum, broadcasts podent afectar communicació, etc.
% tampoc permeten compatibilitat amb lorawan
% \section{Solucions existents}
% loramesher -> sense broadcast, però requereixen descobriment de rutes previs
% Cap dels dos permeten baix consum i modes de deep sleep

% Aqui van els capitols especifics del treball
\chapter{Disseny} % no es veu afectat per material
\section{Elecció del material}
\section{Protocol de comunicació}
\subsection{Accés al medi}
\subsection{Encaminament}
\subsection{Transport}
\subsection{Aplicació}
\section{Reducció del consum}

\chapter{Implementació tècnica}
\section{Estructura}
\subsection{Organització del codi}
% S'ha implementat seguint model de capes, on cada capa és un mòdul.
% Cada capa inferior notifica a la superior a través de callbacks, 
% i cada capa superior notifica a la inferior a través de mètodes.
% S'ha intentat que no hi hagi mètodes bloquejants, amb excepció dels 
% mètodes "finals" com transmissió.
\subsection{Biblioteques i entorn de desenvolupament}
\section{Protocol de comunicació}
\subsection{Accés al medi}
\subsection{Encaminament}
\subsection{Transport}
\subsection{Aplicació}
\section{Aplicació de baix consum}
\section{Validació del funcionament i resultats}


\chapter{Conclusions}
% En el capítol de conclusions cal resumir els resultats del treball. Cal emfasitzar els
% resultats més importants i relacionar-los amb els objectius que es tenien.
% També és l’apartat escaient per indicar els objectius que no s’han assolit i reflexionar
% sobre per què no s’han assolit.
\chapter{Treball futur}
% Si cal es pot afegir un capítol sobre «treball futur» en el que es desgranin aquelles línies
% de treball que podrien ser continuació natural del treball. És convenient exposar, com
% a molt, només les tres o quatre més importants. S’han d’explicar, situar en el marc del
% treball i suggerir breument algunes idees sobre com es podrien encarar


\printbibliography

% Si feu servir apèndixs, descomenteu
% (també la  \part del principi del document)
%\appendix
%\part{Apèndixs}
%\chapter{Un apèndix}


\end{document}

%%% Local Variables:
%%% mode: latex
%%% TeX-master: t
%%% LaTeX-biblatex-use-Biber: t
%%% End:
